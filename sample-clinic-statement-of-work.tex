%%% A Sample Clinic Work Statement Written Using LaTeX.
%%%
%%% C.M. Connelly <cmc@math.hmc.edu>
%%%
%%%  $Id$

%%% Copyright (C) 2004-2005 Claire M. Connelly and 
%%% the Department of Mathematics, Harvey Mudd College.
%%%
%%% This file is part of the sample thesis document provided to HMC
%%% mathematics students.
%%%
%%% See the COPYING document, which should accompany this
%%% distribution, for information about distribution and modification
%%% of the document and its components.


%%% The top part of your document is called the preamble.  You supply
%%% some basic information about the document (such as its title and
%%% author) in a form that LaTeX can understand here.

%%% You can also load additional LaTeX packages, or style files, that
%%% affect the way that the document is laid out, typeset, or supply
%%% additional commands or environments.

%%% The preamble can also be used to define your own commands and
%%% environments, set some constants that will be used throughout your
%%% document, and so on.

%%% As you may have guessed, LaTeX's comment character is the percent
%%% sign.  Any line that starts with a % will be ignored.  You can
%%% also use the comment character to add comments to the end of a
%%% line that will be parsed by TeX.


%%% The first active line in your LaTeX document is the \documentclass
%%% command, which loads a LaTeX class file.  Class files generally
%%% define the appearance of a document, and include a variety of
%%% structural commands.

%%% Clinic reports use the clinic class, which should be located
%%% somewhere in TeX's search path.

%%% For your ``statement of work'' (or ``work statement''), specify
%%% the ``proposal'' document-class option to the hmcclinic class.
\documentclass[proposal]{hmcclinic}

%%% The major difference between the statement of work and a midyear
%%% or final report is that the statement of work is typeset as an
%%% article, which means that the highest level of structural
%%% division available to you is section rather than chapter.

%%% There are also some changes in pagination styles and content
%%% that reflect the briefer nature of the proposal.  For example,
%%% in the longer reports, you use \frontmatter, \mainmatter, and
%%% \backmatter to separate some sections of the report from
%%% others.  In the statement of work, you don't need those
%%% commands, as no such division is necessary.

%%% Other packages needed by your document may be loaded here.
\usepackage{url}              % For formatting URLs and other web or
                              % file references.
\usepackage{mflogo}           % Provides the METAFONT logo.
\usepackage{booktabs}         % Publication-quality tables.
\usepackage{natbib}           % Provides some nice citation and
                              % bibliography formatting commands.
\bibpunct[:~]{(}{)}{;}{a}{,}{,~} % Set some defaults for bibliographic
                                 % punctuation used by natbib.sty.
\usepackage{verbatim}
\usepackage{graphicx}
\usepackage{calc}
\usepackage{subfig}
\usepackage{textcomp}
\usepackage[plainpages=false,pdfpagelabels]{hyperref}


%%% Provide additional context around errors. 
\setcounter{errorcontextlines}{1000}


%%% Information about this document.

%%% I find it most useful to put identifying information about a
%%% document near the top of the preamble.  Technically, this
%%% information must precede the \maketitle command, which often
%%% appears immediately after the beginning of the document 
%%% environment.  Placing it near the top of the document makes it
%%% easier to identify the document, and keeps it out from getting
%%% mixed up with the real meat of the document.

%%% We use the same set of commands for specifying information about
%%% the people involved with the project that are used in the longer
%%% reports, so you can copy most of this information directly into
%%% your midyear and final reports.

%%% So, some questions.

%% What is the name of the company or organization sponsoring your project?
\sponsor{Harvey Mudd College}

%% What is the title of your report?
\title{A Sample Clinic Work Statement}

%% Who are the authors of the report (your team members)?  (Separate
%% names with \and.)
\author{Claire Connelly (Project Manager) \and Melissa O'Neill}

%% What is your faculty advisor's name?  (Again, separate names with
%% \and, if necessary.)
\advisor{Melissa O'Neill}

%% Liaison's name or names?
\liaison{Sam Neal}

%% Did you have an outside consultant help you with this project?  Put
%% their names in the \consultant command.
\consultant{Joseph Jones}

%%% End of information section.

%%% New commands and environments.

%%% You can define your own commands and environments here.  If you
%%% have a lot of material here, you might want to consider splitting
%%% the commands and environments into a separate ``style'' file that
%%% you load with \usepackage.

\newcommand{\coolcommand}[1]{#1 is cool.} % Lets everyone know that
                                % the person or thing that you provide
                                % as the argument to the command is
                                % cool.

%%% You probably won't want any of the following commands, which are
%%% here to allow various the names of commands, make examples typeset
%%% properly, and so on.  You can, of course, use them as examples for
%%% your own user-defined commands.
                                                                               
\newcommand{\bslash}{\symbol{'134}}%backslash
\newcommand{\bsl}{{\texttt{\bslash}}}
\newcommand{\com}[1]{\bsl\texttt{#1}\xspace}
\newcommand{\file}[1]{\texttt{#1}\xspace}

\newcommand{\pdftex}{PDF\tex}
\newcommand{\pdflatex}{PDF\latex}
\newcommand{\acronym}[1]{\textsc{#1}\xspace}
\newcommand{\key}[1]{\textsf{\emph{#1}}\xspace}
\newcommand{\class}[1]{\textsf{#1}\xspace}
\newcommand{\package}[1]{\textsf{#1}\xspace}
\newcommand{\env}[1]{\texttt{#1}\xspace}
\newcommand{\prog}[1]{\texttt{#1}\xspace}
\newcommand{\command}[1]{\texttt{\bsl{}#1}\xspace}
\newcommand{\ctt}{\texttt{comp.text.tex}\xspace}
\newcommand{\tex}{\TeX\xspace}
\newcommand{\latex}{\LaTeX\xspace}
\newcommand{\host}[1]{\textsf{#1}\xspace}
                                                                                


\newcounter{cms}

%%% Some theorem-like command definitions.

%%% The \newtheorem command comes from the amsthm package.  That
%%% package is loaded by the class file.

%%% Note that these definitions have changed from the version in the
%%% sample report document by dropping the ``within'' argument.  See
%%% Gratzer's _Math into LaTeX_ or the AMS-LaTeX documentation for
%%% more details.

\newtheorem{thm}{Theorem}
\newtheorem{Theo1}{Theorem}
\newtheorem{Theo2}{Theorem}
\newtheorem{Lemma}{Lemma}


%%% If you find that some words in your document are being hyphenated
%%% incorrectly, you can specify the correct hyphenation using the
%%% \hyphenation command.  Note that words are separated by
%%% whitespace, as shown below.

\hyphenation{ap-pen-dix wer-ther-i-an}


%%% The start of the document!

%% The document environment is the main environment in any LaTeX
%% document.  It contains other environments, as well as your text.

\begin{document}

%%% The front matter of a large document includes the title page or
%%% pages, tables of contents, lists of figures or tables, and so on,
%%% your abstract, a preface or introduction, and so on.  It's
%%% delineated with the \frontmatter command.
\frontmatter

%%% One of the things that the \frontmatter does is make page
%%% numbers appear as lowercase Roman numerals---i, vi, xii, and so
%%% on.

%%% The first thing in the front matter is your title page.  The title
%%% page is formatted by commands in the document class file, so you
%%% don't need to worry about what it looks like -- just putting the
%%% \maketitle command in your document (and filling in the necessary
%%% information for the identification commands above) is enough.
\maketitle


%%% Abstract

%%% Your abstract should be a \emph{brief} summary of the contents
%%% of your report.  Don't go into excruciating detail
%%% here---there's plenty of room for that later.

%%% If possible, limit your abstract to a single paragraph, as your
%%% abstract may be used in promotional materials for the Clinic.
\begin{abstract}
  This sample document and the \textsf{hmcclinic} class were
  created for the Harvey Mudd College Computer Science and Mathematics
  Clinics by Claire Connelly and Melissa O'Neill.
\end{abstract}




%%% Table of Contents, List of Figures, and List of Tables.
%%% 
%%% If you don't have any figures or tables in your report, you can
%%% comment out the appropriate command.

\tableofcontents
\listoffigures
\listoftables


%%% End of the front matter.

%%% Beginnning of the main matter.

%% The main part of your report consists of normal, numbered
%% chapters.  The main matter is opened with the \mainmatter command.
\mainmatter


%%% Content.

%%% For smaller documents---especially those you're writing by
%%% yourself---you might write your entire report using a single LaTeX
%%% source file.  For larger documents, we recommend that you split
%%% the source file into several separate, smaller files.  The smaller
%%% files are ``included'' into your main, or ``master'' document
%%% using \include commands.

%%% Splitting your source has several advantages.  One, it allows you
%%% to have more than one person working on different parts of the
%%% document at the same time (although we still recommend that you
%%% use CVS or a similar revision-control system!).  Two, smaller
%%% document chunks allow you to reorganize your document more easily.
%%% If your decide that Chapter 8 would be better as Chapter 4, all
%%% you have to do is swap the \include commands around.  For that
%%% reason, you may want to consider giving your separate chapters
%%% meaningful names rather than calling them ``chapter1'',
%%% ``chapter2'', and so on.

%%% Finally, splitting the document allows you to concentrate on a
%%% particular section without being distracted by other
%%% sections---all you have to do is comment out the \include line for
%%% the sections you're not working on.  This technique can be
%%% especially useful when you're trying to track down a problem by
%%% allowing you to easily locate the file with the problem and
%%% rule out the other sections.

%%% The sample ``statement of work'' typesets as an article, so you
%%% can either just write the whole thing as one document or use
%%% the \include command to include separate files for major
%%% sections.  (Even though you're using sections rather than
%%% chapters, the document still consists of several different
%%% structural sections that you can write and edit separately to
%%% create your final document.)

%%% If you want to use separate files for different text sections,
%%% look at the sample report document to see how to use
%%% the \includeonly and \include commands.  (\includeonly is, of
%%% course, optional---you're welcome to typeset everything you have
%%% every time---but you may find it useful for any of several
%%% reasons.)






%%% Appendices.

%%% Appendices are just like chapters, only they're generally lettered
%%% rather than numbered (although that depends on your document
%%% class, of course).

%%% The appendices are delineated with the \appendix command.
%%% Individual appendices are begun with the standard \chapter or
%%% \section commands.  In our example, we'll \include them just as we
%%% did other chapters.

%%% Even in a relatively short document such as your statement of
%%% work, you might need to have appendices.  If so, uncomment
%%% the \appendix command and add them below (remember, the
%%% top-level structural command in this format is section).

% \appendix


%%% Bibliography.

%%% BibTeX is the tool to use for citations and layout of your
%%% bibliography.  Instead of having to type ``[5]'' or ``(Jones,
%%% 1968)'' (and keep track of which citation is which and renumber
%%% them as you add more references to your bibilography), you use
%%% special commands that allow BibTeX and LaTeX to automatically put
%%% the correct information in the right place.

%%% Depending on your field, it may or may not be appropriate to list
%%% references for which you haven't included specific citations.  If
%%% your field sanctions such practices, or if you just want to get an
%%% idea of what you have in your bibliography file, you can include
%%% everything with the \nocite{*} command.
\nocite{*} 


%%% The appearance of your bibliography and citations in your text are
%%% defined by a combination of any bibliography-related LaTeX
%%% packages (such as natbib, harvard, or chicago) and the particular
%%% bibliography style file that you load with the \bibliographystyle
%%% command.  Bibliography-style files end in .bst; you can find them
%%% by searching your file system using whatever tools you have for
%%% doing searches.  (On most modern Unices, ``locate .bst'' will give
%%% you an idea of what's available.)

\bibliographystyle{plainnat}

%%% The particular bibliography data file or files that you want to
%%% use are specified with the \bibliography file.  Multiple files are
%%% separated by commas.

%%% You might want to use multiple bibliography (or ``bib'') files if
%%% you had a master bib file containing references you use again and
%%% again, and another containing only records for references for a
%%% particular project.

%%% Many people create a single, large bib file that they use for
%%% everything they write.  That approach requires you to \cite every
%%% reference that you want to use in your document -- using
%%% \nocite{*} with a huge bibliography database will give you a large
%%% bibliography containing many references you haven't consulted for
%%% your particular document!

\bibliography{sample}


%%% Glossary or Index.

%%% Having a glossary or index in a statement of work is overkill.
%%% Just define your terms in the text and you'll be fine.

\end{document}


